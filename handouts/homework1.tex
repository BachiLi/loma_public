\documentclass{article}
\usepackage{hyperref}
\usepackage{listings}
\usepackage{color}
\usepackage{xcolor}
\usepackage{geometry}
\usepackage{graphicx}
\usepackage{amsmath}
\usepackage{caption}
\usepackage{subcaption}
\usepackage{syntax}
\geometry{margin=1in}
\pdfminorversion=6

\newcommand\TODO[1]{\textcolor{red}{TODO: #1}}

\newcommand\header[2]{
    \begin{center}
        {\large
        UCSD CSE 291 (Differentiable Programming) Assignment #1: \\
        \vspace{0.3cm}
        \Large
        #2}
    \end{center}
}

\definecolor{dkgreen}{rgb}{0,0.6,0}
\definecolor{gray}{rgb}{0.5,0.5,0.5}
\definecolor{mauve}{rgb}{0.58,0,0.82}
\lstset{frame=tb,
        aboveskip=3mm,
        belowskip=3mm,
        showstringspaces=false,
        columns=flexible,
        basicstyle={\small\ttfamily},
        numbers=none,
        numberstyle=\tiny\color{gray},
        keywordstyle=\color{blue},
        commentstyle=\color{dkgreen},
        stringstyle=\color{mauve},
        breaklines=true,
        breakatwhitespace=true,
        tabsize=2
}

\hypersetup{colorlinks=true}


\begin{document}

\header{1}{Forward mode automatic differentiation}

In our first homework, we will implement what is known as the ``forward mode'' automatic differentiation in loma for generating code that computes the derivatives. 
Specifically, given a (differentiable) function $f: \mathbb{R}^n \rightarrow \mathbb{R}^m$, we will automatically generate its \href{https://en.wikipedia.org/wiki/Total_derivative}{total derivative} $df: \mathbb{R}^n \times \mathbb{R}^n \rightarrow \mathbb{R}^m$. 
The total derivative $df(\mathbf{x}, \mathbf{dx})$ (noticed that it takes two inputs $\mathbf{x}$ and $\mathbf{dx}$ instead of only one!) is defined as the best linear approximation at point $\mathbf{x}$:
\begin{equation}
\lim_{dx \rightarrow 0} \frac{\left\|f(x + dx) - \left(f(x) + df(x, dx)\right)\right\|}{\left\|dx\right\|} = 0,
\label{eq:totalderiv}
\end{equation}
where $df(x, dx)$ is linear over the $dx$ argument (but not necessarily linear over the $x$ argument).

For notational convienence, we further define a function $Df: \mathbf{R}^n \times \mathbf{R}^n \rightarrow \mathbb{R}^m \times \mathbb{R}^m$ that combines the outputs of function $f$ and its total derivative $df$:
\begin{equation}
Df(x, dx) = \left(f(x), df(x, dx)\right).
\end{equation}

We will focus on straight-line code for this homework, that is, code without if-else statements, loops, and function calls to other loma functions. We will relax this assumption in Homework 3. 

The key idea of forward-mode automatic differentiation is the \textbf{chain rule}. Let
\begin{equation}
	f(x) = g(h(x)).
\end{equation}
Then,
\begin{equation}
	Df(x, dx) = Dg(Dh(x, dx)).
\end{equation}
(Try proving this using the definition ot total derivative!)

Therefore, if we have a straight-line code of a function \lstinline{f}
\begin{lstlisting}[language=Python]
def f(x : in[float], y : in[float]) -> float:
	z0 : float = h(x, y)
	z1 : float = g(z0)
	return z1
\end{lstlisting}
We can automatically synthesize its \lstinline{Df} by 1) replacing all datatypes \lstinline{float} with a new struct \lstinline{DFloat} that stores both the primal value and the total derivative, and 2) replacing all function calls with their own ``Df'':
\begin{lstlisting}[language=Python]
class DFloat:
	val : float
	dval : float

def Df(x : in[DFloat], y : in[DFloat]) -> DFloat:
	z0 : DFloat = Dh(x, y)
	z1 : DFloat = Dg(z0)
	return z1
\end{lstlisting}
We can recursively perform this transformation to function \lstinline{h} and \lstinline{g}, then we are done!

Now, we need to provide the terminal cases of our recursive transformation. Notice that these \lstinline{h} and \lstinline{g} functions can be \emph{any} function. For example, if $h(x, y) = x + y$, then we can derive (from the definition of total derivative) that $Dh(x, y, dx, dy) = (x + y, dx + dy)$. 

Here, we provide the list of terminal cases of the basic operations in loma:
\begin{equation}
\begin{aligned}
\text{DAdd}(x, y, dx, dy) &= (x + y, dx + dy) \\
\text{DSub}(x, y, dx, dy) &= (x - y, dx - dy) \\
\text{DMul}(x, y, dx, dy) &= (x \cdot y, x \cdot dy + y \cdot dx) \\
\text{DDiv}(x, y, dx, dy) &= \left(\frac{x}{y}, \frac{x \cdot dy - y \cdot dx}{y^2}\right) \\
\text{DSin}(x, dx) &= \left(\sin(x), \cos(x) \cdot dx\right) \\
\text{DCos}(x, dx) &= \left(\cos(x), -\sin(x) \cdot dx\right) \\
\text{DSqrt}(x, dx) &= \left(\sqrt{x}, \frac{dx}{2\sqrt{x}}\right) \\
\text{DPow}(x, y, dx, dy) &= \left(x^y, dx \cdot \left(y \cdot x^{y-1}\right) + dy \cdot \left(x^y \log(x)\right)\right) \\
\text{DExp}(x, dx) &= \left(\exp{x}, \exp{x} \cdot dx\right) \\
\text{DLog}(x, dx) &= \left(\log{x}, \frac{dx}{x}\right) \\
\end{aligned}
\end{equation}



\end{document}