\documentclass{article}
\usepackage{hyperref}
\usepackage{listings}
\usepackage{color}
\usepackage{xcolor}
\usepackage{geometry}
\usepackage{graphicx}
\usepackage{amsmath}
\usepackage{amsfonts}
\usepackage{caption}
\usepackage{subcaption}
\usepackage{syntax}
\geometry{margin=1in}
\pdfminorversion=6

\newcommand\TODO[1]{\textcolor{red}{TODO: #1}}

\newcommand\header[2]{
    \begin{center}
        {\large
        UCSD CSE 291 (Differentiable Programming) Assignment #1: \\
        \vspace{0.3cm}
        \Large
        #2}
    \end{center}
}

\definecolor{dkgreen}{rgb}{0,0.6,0}
\definecolor{gray}{rgb}{0.5,0.5,0.5}
\definecolor{mauve}{rgb}{0.58,0,0.82}
\lstset{frame=tb,
        aboveskip=3mm,
        belowskip=3mm,
        showstringspaces=false,
        columns=flexible,
        basicstyle={\small\ttfamily},
        numbers=left,
        numberstyle=\tiny\color{gray},
        keywordstyle=\color{blue},
        commentstyle=\color{dkgreen},
        stringstyle=\color{mauve},
        breaklines=true,
        breakatwhitespace=true,
        tabsize=2
}

\hypersetup{colorlinks=true}


\begin{document}

\header{1}{Forward mode automatic differentiation}

\section{Total derivatives}

In our first homework, we will implement what is known as the ``forward mode'' automatic differentiation in loma for generating code that computes the derivatives. 
Specifically, given a (differentiable) function $f: \mathbb{R}^n \rightarrow \mathbb{R}^m$, we will automatically generate its \href{https://en.wikipedia.org/wiki/Total_derivative}{total derivative} $df: \mathbb{R}^n \times \mathbb{R}^n \rightarrow \mathbb{R}^m$. 
The total derivative $df(\mathbf{x}, \mathbf{dx})$ (noticed that it takes two inputs $\mathbf{x}$ and $\mathbf{dx}$ instead of only one!) is defined as the best linear approximation at point $\mathbf{x}$:
\begin{equation}
\lim_{\mathbf{dx} \rightarrow 0} \frac{\left\|f(\mathbf{x} + \mathbf{dx}) - \left(f(\mathbf{x}) + df(\mathbf{x}, \mathbf{dx})\right)\right\|}{\left\|\mathbf{dx}\right\|} = 0,
\label{eq:totalderiv}
\end{equation}
where $df(\mathbf{x}, \mathbf{dx})$ is linear over the $\mathbf{dx}$ argument (but not necessarily linear over the $x$ argument).

We can get the partial or directional derivatives that we are more used to from the total derivative function $df$. For example, we can apply a ``one-hot'' vector $\mathbf{v}_i$ 
\begin{equation}
\frac{\partial}{\partial x_i} f(\mathbf{x}) = df(\mathbf{x}, \mathbf{v}_i), v_{i,j} = \begin{cases}
1 & \text{ if } i = j, \\
0 & \text{ otherwise}.
\end{cases}
\end{equation}

For notational convienence, we further define a function $Df: \mathbf{R}^n \times \mathbf{R}^n \rightarrow \mathbb{R}^m \times \mathbb{R}^m$ that combines the outputs of function $f$ and its total derivative $df$:
\begin{equation}
Df(x, dx) = \left(f(x), df(x, dx)\right).
\end{equation}

We will focus on straight-line code for this homework, that is, code without if-else statements, loops, and function calls to other loma functions. We will relax this assumption in Homework 3. 

The key idea of forward-mode automatic differentiation is the \textbf{chain rule}. Let
\begin{equation}
	f(x) = g(h(x)).
\end{equation}
Then,
\begin{equation}
	Df(x, dx) = Dg(Dh(x, dx)).
\end{equation}
(Try proving this using the definition ot total derivative!)

Therefore, if we have a straight-line code of a function \lstinline{f}
\begin{lstlisting}[language=Python]
def f(x : in[float], y : in[float]) -> float:
	z0 : float = h(x, y)
	z1 : float = g(z0)
	return z1
\end{lstlisting}
We can automatically synthesize its \lstinline{Df} by 1) replacing all datatypes \lstinline{float} with a new struct \lstinline{DFloat} that stores both the primal value and the total derivative, and 2) replacing all function calls with their own ``Df'':
\begin{lstlisting}[language=Python]
class DFloat:
	val : float
	dval : float

def Df(x : in[DFloat], y : in[DFloat]) -> DFloat:
	z0 : DFloat = Dh(x, y)
	z1 : DFloat = Dg(z0)
	return z1
\end{lstlisting}
We can recursively perform this transformation to function \lstinline{h} and \lstinline{g}, then we are done!

Now, we need to provide the terminal cases of our recursive transformation. Notice that these \lstinline{h} and \lstinline{g} functions can be \emph{any} function. For example, if $h(x, y) = x + y$, then we can derive (from the definition of total derivative) that $Dh(x, y, dx, dy) = (x + y, dx + dy)$. 

Here, we provide the list of terminal cases of the basic operations in loma:
\begin{equation}
\begin{aligned}
\text{DConstant}(c) &= (c, 0) \\
\text{DVariable}(x, dx) &= (x, dx) \\
\text{DAdd}(x, y, dx, dy) &= (x + y, dx + dy) \\
\text{DSub}(x, y, dx, dy) &= (x - y, dx - dy) \\
\text{DMul}(x, y, dx, dy) &= (x \cdot y, x \cdot dy + y \cdot dx) \\
\text{DDiv}(x, y, dx, dy) &= \left(\frac{x}{y}, \frac{x \cdot dy - y \cdot dx}{y^2}\right) \\
\text{DSin}(x, dx) &= \left(\sin(x), \cos(x) \cdot dx\right) \\
\text{DCos}(x, dx) &= \left(\cos(x), -\sin(x) \cdot dx\right) \\
\text{DSqrt}(x, dx) &= \left(\sqrt{x}, \frac{dx}{2\sqrt{x}}\right) \\
\text{DPow}(x, y, dx, dy) &= \left(x^y, dx \cdot \left(y \cdot x^{y-1}\right) + dy \cdot \left(x^y \log(x)\right)\right) \\
\text{DExp}(x, dx) &= \left(\exp{x}, \exp{x} \cdot dx\right) \\
\text{DLog}(x, dx) &= \left(\log{x}, \frac{dx}{x}\right) \\
\text{Dfloat2int}(x, dx) &= \left(\text{float2int}(x), 0\right)
\end{aligned}
\end{equation}

\section{Forward-mode automatic differentiation in loma}

Our goal is this homework is to transform loma functions to their derivatives. In loma, the total derivative of a function is declared as follows:
\begin{lstlisting}[language=Python]
def f(x : float) -> float:
	# ...

df = fwd_diff(f) # df has type _dfloat -> _dfloat
\end{lstlisting}

Loma will automatically generate the content of \lstinline{df}, so the programmer does not have to implement it themselves.

The transformation will happen in \lstinline{autodiff.py}. \textbf{Your task is to fill in the blanks in the \lstinline{forward_diff} function. Use \lstinline{hw_tests/test.py} to test your implementation. If you pass all tests, you get full points. Each test is weighted equally.}

\subsection{Transformation of types}
We (well, I) have implemented most of this part for you, but you should still understand this part for your implementation. We define two basic types for differentiation: \lstinline{_dfloat} and \lstinline{_dint}:
\begin{lstlisting}[language=Python]
class _dfloat:
	val : float
	dval : float

class _dint:
	val : int
\end{lstlisting}
These types represent the primal values and total derivatives of floats and integers (for integers, the differential is always zero). For a general struct with id \lstinline{X}, we synthesize a new differential struct with id \lstinline{_dX}, where we convert the types of all its members to the differential type. For example, for the following types:
\begin{lstlisting}[language=Python]
class Foo:
	x : int
	y : float
	z : Bar

class Bar:
	a : float
	b : int
\end{lstlisting}
We will generate the following new types to represent the primal values and total derivatives:
\begin{lstlisting}[language=Python]
class _dFoo:
	x : _dint
	y : _dfloat
	z : _dBar

class _dBar:
	a : _dfloat
	b : _dint
\end{lstlisting}

For a function definition, we transform all argument types and return type to their \lstinline{_d} counterparts. For example, for the following function \lstinline{f}:
\begin{lstlisting}[language=Python]
def f(x : float, y : Foo) -> Bar:
	# ...

df = fwd_diff(f)
\end{lstlisting}
\lstinline{df} would have the type:
\begin{lstlisting}[language=Python]
def df(x : _dfloat, y : _dFoo) -> _dBar:
	# ...
\end{lstlisting}

Finally, loma introduces a type modifier \lstinline{Diff[]} that allows one to specify the \lstinline{_d} counterpart of a type. For example, \lstinline{Diff[float]} resolves to the type \lstinline{_dfloat}, and \lstinline{Diff[Foo]} resolves to the type \lstinline{_dFoo}. 

We have implemented these transformations except for the function definition part in \lstinline{resolve_diff_types()} in \lstinline{autodiff.py}.

\subsection{Transformation of statements and expressions}
This is the main thing you will need to figure out for this homework. We will use an \lstinline{IRMutator} to visit a \lstinline{FunctionDef} node that represents the primal function, and transform it to the derivative function. I recommend you to try to pass the tests one-by-one and incrementally build up your implementation. Here, we briefly describe each test and what you need to implement to pass the tests.

\paragraph{test_identity} The test asks you to transform an identity function.
\begin{lstlisting}[language=Python]
def identity(x : In[float]) -> float:
    return x

d_identity = fwd_diff(identity)
\end{lstlisting}
\lstinline{d_identity} should be something like:
\begin{lstlisting}[language=Python]
def d_identity(x : In[_dfloat]) -> _dfloat:
	_ret : _dfloat
	_ret.val = x.val
	_ret.dval = x.dval
	return _ret
\end{lstlisting}
You will need to implement \lstinline{mutate_function_def}, \lstinline{mutate_return}, and \lstinline{mutate_var} for the transformation. You may find the \lstinline{type_to_diff_type} useful for transforming the function signatures. I also find it useful to use the \lstinline{PrimalMutator} to convert access to \lstinline{x} to \lstinline{x.val} (you will need to implement the primal mutator yourself).

\paragraph{test_constant} This time, you are asked to transform a constant function.
\begin{lstlisting}[language=Python]
def constant(x : In[float]) -> float:
    return 2.0

d_constant = fwd_diff(constant)
\end{lstlisting}
\lstinline{d_constant} should be something like:
\begin{lstlisting}[language=Python]
def d_constant(x : In[_dfloat]) -> _dfloat:
	_ret : _dfloat
	_ret.val = 2.0
	_ret.dval = 0.0
	return _ret
\end{lstlisting}
You will need to implement \lstinline{mutate_const_float} to pass this test.

\paragraph{test_binary_ops} This test asks you to implement the derivatives of some binary operations:
\begin{lstlisting}[language=Python]
def plus(x : In[float], y : In[float]) -> float:
    return x + y

def subtract(x : In[float], y : In[float]) -> float:
    return x - y

def multiply(x : In[float], y : In[float]) -> float:
    return x * y

def divide(x : In[float], y : In[float]) -> float:
    return x / y

d_plus = fwd_diff(plus)
d_subtract = fwd_diff(subtract)
d_multiply = fwd_diff(multiply)
d_divide = fwd_diff(divide)
\end{lstlisting}
To pass this test, implement \lstinline{mutate_add}, \lstinline{mutate_sub}, \lstinline{mutate_mul} and \lstinline{mutate_div}. From here, the \lstinline{PrimalMutator} will become more useful. You will need it to convert access to \lstinline{x} and \lstinline{y} to \lstinline{x.val} and \lstinline{y.val}.

\paragraph{test_declare} Now we will start to have variable declaration:
\begin{lstlisting}[language=Python]
def declare(x : In[float], y : In[float]) -> float:
    z0 : float = x + y
    z1 : float = z0 + 5.0
    z2 : float = z1 * z0
    return z2

d_declare = fwd_diff(declare)
\end{lstlisting}
You'll need to implement \lstinline{mutate_declare} to pass this test.

\paragraph{test_assign and test_side_effect} Next, we test variable assignment.
\begin{lstlisting}[language=Python]
def assign(x : In[float], y : In[float]) -> float:
    z : float
    z = x + y
    return z

d_assign = fwd_diff(assign)
\end{lstlisting}

\begin{lstlisting}[language=Python]
def side_effect(x : In[float], y : In[float]) -> float:
    z : float
    z = x + y
    z = 0.0
    z = x * y
    return z

d_side_effect = fwd_diff(side_effect)
\end{lstlisting}

You'll need to implement \lstinline{mutate_assign} to pass these tests.

\paragraph{test_call} Next, we test intrinsic function calls (you don't need to consider calls to other loma functions yet).
\begin{lstlisting}[language=Python]
def call(x : In[float]) -> float:
    z0 : float = sin(x)
    z1 : float = cos(z0) + 1.0
    z2 : float = sqrt(z1)
    z3 : float = pow(z2, z1)
    z4 : float = exp(z3)
    z5 : float = log(z3 + z4)
    return z5

d_call = fwd_diff(call)
\end{lstlisting}
You'll need to implement \lstinline{mutate_call} to pass this test.

\paragraph{test_int_input and test_int_input} Next, we test the handling of integers.
\begin{lstlisting}[language=Python]
def int_input(x : In[float], y : In[int]) -> float:
    z : int = 5
    return z * x + y - 1

d_int_input = fwd_diff(int_input)
\end{lstlisting}
\begin{lstlisting}[language=Python]
def int_output(x : In[float], y : In[int]) -> int:
    z : int = 5
    return z * x + y - 1

d_int_output = fwd_diff(int_output)
\end{lstlisting}
You'll need to go back to your previous code, and make sure that you handle integers correctly. In particular, \lstinline{_dint} does not have a \lstinline{dval} member, so you need to handle this.

\paragraph{test_array_output, test_array_input, test_int_array_input, test_array_input_indexing, test_array_output_indexing, multiple_outputs} Next, we test the handling of arrays.
\begin{lstlisting}[language=Python]
def array_output(x : In[float], y : Out[Array[float]]):
    y[0] = x * x
    y[1] = x * x * x

d_array_output = fwd_diff(array_output)
\end{lstlisting}
\begin{lstlisting}[language=Python]
def array_input(x : In[Array[float]]) -> float:
    return x[0] + x[1]

d_array_input = fwd_diff(array_input)
\end{lstlisting}
\begin{lstlisting}[language=Python]
def int_array_input(x : In[Array[float]], y : In[Array[int]]) -> float:
    return x[0] + x[1] + y[0]

d_int_array_input = fwd_diff(int_array_input)
\end{lstlisting}
\begin{lstlisting}[language=Python]
def array_input_indexing(x : In[Array[float]], i : In[int], j : In[float]) -> float:
    return x[i] + x[float2int(j)] + x[2 * i] + x[2 * float2int(j)]

d_array_input_indexing = fwd_diff(array_input_indexing)
\end{lstlisting}
\begin{lstlisting}[language=Python]
def array_output_indexing(x : In[float], i : In[int], j : In[float], y : Out[Array[float]]):
    y[i] = x
    y[float2int(j)] = 2 * x
    y[2 * i] = 3 * x
    y[2 * float2int(j)] = 4 * x

d_array_output_indexing = fwd_diff(array_output_indexing)
\end{lstlisting}
\begin{lstlisting}[language=Python]
def multiple_outputs(x : In[float], y : Out[Array[float]], z : Out[Array[float]]):
    y[0] = x * x
    y[1] = x * x * x
    z[0] = y[0] * y[1]

d_multiple_outputs = fwd_diff(multiple_outputs)
\end{lstlisting}
You'll need to implement \lstinline{mutate_array}, and go back to your previous code to make sure you handle array types properly. You also need to handle the \lstinline{float2int} function call.

\paragraph{test_struct_input, test_nested_struct_input, test_struct_output, test_nested_struct_output, test_array_in_struct, test_struct_in_array} Next, we test the differentiation with Structs.
\begin{lstlisting}[language=Python]
class Foo:
    x : float
    y : int

def struct_input(foo : In[Foo]) -> float:
    z : int = 5
    return z * foo.x + foo.y - 1

d_struct_input = fwd_diff(struct_input)
\end{lstlisting}
\begin{lstlisting}[language=Python]
class Foo:
    x : float
    y : Bar

class Bar:
    z : int
    w : float

def nested_struct_input(foo : In[Foo]) -> float:
    b : Bar
    b.z = 5
    b.w = 3
    return (foo.x + foo.y.z + foo.y.w + b.z) * b.w

d_nested_struct_input = fwd_diff(nested_struct_input)
\end{lstlisting}
\begin{lstlisting}[language=Python]
class Foo:
    a : float
    b : int

def struct_output(x : In[float], y : In[int]) -> Foo:
    foo : Foo
    foo.a = x + y * x
    foo.b = y - x
    return foo

d_struct_output = fwd_diff(struct_output)
\end{lstlisting}
\begin{lstlisting}[language=Python]
class Foo:
    int_arr : Array[int]
    float_arr : Array[float]

def array_in_struct(f : In[Foo]) -> float:
    return f.int_arr[0] + f.float_arr[0]

d_array_in_struct = fwd_diff(array_in_struct)
\end{lstlisting}
\begin{lstlisting}[language=Python]
class Foo:
    x : int
    y : float

def struct_in_array(in_f : In[Array[Foo]], out_f : Out[Array[Foo]]):
    out_f[0].x = in_f[0].x * in_f[0].y
    out_f[0].y = in_f[0].x + in_f[0].y

d_struct_in_array = fwd_diff(struct_in_array)
\end{lstlisting}
You will need to implement \lstinline{mutate_struct_access}, and revisit other part of your code to make sure you have handled Struct correctly.

\paragraph{test_poly} The last test shows how differentiated code can be used in loma code:
\begin{lstlisting}[language=Python]
def poly(x : In[float]) -> float:
    return 3 * x * x * x * x + 5 * x * x + 10

d_poly = fwd_diff(poly)

def d_poly_dx(x : In[float]) -> float:
    d_x : Diff[float]
    d_x.val = x
    d_x.dval = 1.0
    return d_poly(d_x).dval
\end{lstlisting}
You probably don't need to implement anything extra to pass this test.

\end{document}
